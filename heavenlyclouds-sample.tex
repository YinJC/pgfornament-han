% !TEX program=xelatex
% \documentclass[aspectratio=169]{ctexbeamer}
\documentclass{ctexbeamer}

\usetheme{HeavenlyClouds}

% Mac 或 Linux 上请自己设定 \lishu!!
% \providecommand{\lishu}{\CJKfontspec{Libian SC}[BoldFont=Baoli SC,Scale=1.2]}

\author{林莲枝}
\title{神马浮云\\Beamer主题}
\subtitle{\texttt{pgfornament-han}附录福利}

% 可以放各自高校的校徽,中国高校校徽好像大多数都是圆形的。确保一下背景是透明色的 png 或 pdf 档就可以了
% \titlegraphic{\includegraphics[width=2cm]{example-grid-100x100pt}}

% \alttitlecircle  %% 试试取消注释,有惊喜

\begin{document}

\begin{frame}[noframenumbering]
  \maketitle
\end{frame}

\section{简介}

\begin{frame}[fragile,allowframebreaks]
  \frametitle{此主题脑洞略大}

  \begin{itemize}
    \item 每一页的背景的云彩纹样,位置、大小、深浅都是随机的
    \item 所以如果内容很多页的话,编译时运算可能会比较花时间
    \item 导言区加上 \verb|\alttitlecircle|,标题页的纹样会变化

    \framebreak
    \item 标题页的寿字纹,可以用各自高校的标志代替。中国高校校徽很多都是圆形的。确保一下背景是透明色的 png 或 pdf 档就可以了

    \verb|\titlegraphic{\includegraphics[width=2cm]{logo}}|

    \medskip

    \item 浮云有了,神马呢?
      \begin{itemize}
        \item 以后如果做出来了马的简易图腾纹样,才考虑取代掉进度条的圆点吧
      \end{itemize}
  \end{itemize}

\end{frame}

\begin{frame}
  \frametitle{套马的汉子你威武雄壮}

  \begin{itemize}
    \item 飞驰的骏马像疾风一样
    \item 一望无际的原野随你去流浪
    \item 你的心海和大地一样宽广
  \end{itemize}

\end{frame}

\begin{frame}[allowframebreaks]
  \frametitle{各种 block}

  \begin{block}{你是我天边最美的云彩}
    让我用心把你留下来
  \end{block}

  \begin{exampleblock}{你是我天边最美的云彩}
    让我用心把你留下来
  \end{exampleblock}

  \begin{alertblock}{你是我天边最美的云彩}
    让我用心把你留下来
  \end{alertblock}

  \begin{theorem}[你是我天边最美的云彩]
    让我用心把你留下来
  \end{theorem}

  \begin{proof}[你是我天边最美的云彩]
    让我用心把你留下来
  \end{proof}
\end{frame}

\end{document}
